\begin{preface}
很早就听说过\LaTeX\index{\LaTeX}了,但却一直没有真正学习过,直到今年,需要处理一些大文档,想起了\LaTeX。
重新翻出\LaTeX的文档,从CCT开始,至于为什么是CCT,
因为Ctex\index{CTeX}提供的那个CTeX FAQ里对中文的第一个例子,就是以CCT
为例写的。
CCT是中科院的张林波研究员写的,帮助文档都是中文,看起来比较容易,但毕竟是好几年前的版本了,
更新也并不是那么及时,而且CCT\index{CCT}早期版本的字体是点阵字体,边缘很粗糙,
虽然不影响打印,但在这个年代还在用着这样的字体,着实不是那么舒服。
我又开始了第二个例子,CJK的尝试,在尝试CJK\index{CJK}的过程中,
无意中看到了CTeX的ctexart,ctexbook和ctexrep这几个基本模版,这才找到CTeX的门,
筒子们不要笑我绕了这么一大圈才摸进了CTeX的门,虽然从开始就使用的是CTeX的发行版。

这里也要说一下,CTeX提供的部分帮助文档内容也比较老了,一些操作现在新的软件虽然仍然兼容,
但已经不是新版软件推荐的做法了,比如,CTeX FAQ里面对于pdf文件的生成,
依然是先由latex.exe生成dvi文件,再由dvi文件生成ps文件,最后再生成pdf文件。
实际上,现在流行的新版TeX类软件都已经将pdfTeX\index{pdfTeX}作为默认引擎,支持直接生成pdf文件,
而且dvi、ps文件的打开速度比pdf反而要慢许多。我使用的是64位系统,CTeX提供的安装包只支持32位系统,
我单独安装的MikTeX\index{MikTeX} x64\index{x64}版使用CTeX模版生成的dvi文件使用dvips\index{dvips}处理时会找不到字体,
因为这个问题,我找了很久,最后的结论是:dvips可以放弃了,直接使用dvipdfm\index{dvipdfm}更合适。

后来几天在\LaTeX的实践中看不少相关细节,开始对其模版产生了兴趣,
在88上TeX版把置顶的ZJUthesis下了下来,就是写这个模版的基础,数学系模版。
下下来后发现这个模板给的例子pdf与当前学校使用的2008年论文模版差别老大了,从封面到目录,
章节格式,都是完全不一样,因此,决定着手做一个与学样提供的Word模版比较接近的模版。

在以2006年数学系模版为基础进行新模版编写的过程中,学了不少方法,
也发现老模版不少过时或者不合适的地方。
第一个学到的就是,从模版一开头就发现这个模版是以ctexbook这个模版为基础制作的,
做到模版完成的时候,
发现88的TeX版置顶模版已经更新,我以为我白做了,
下下来一看,原来这个新的模版不是以ctexbook为基础制作的,而是更基础的\LaTeXe\index{\LaTeX}
对比自己基本完工的模版,才发现ctexbook为我省了很多工作量。只是一些修修改改就做到了很接近学校
word模版的效果。
ctexbook的新版已经直接将hyperref包打了进去,2006年数学系模版对hyperref\index{hyperref}的引用判断部分已经明显示过时,在用新版Miktex运行的时候直接报错了。
在编写封面的时候,发现2006年的模版用了一个五列的表格,可这部分的内容只需要两列就够了,
直到我某天下载了中科院的模版后才明白,2006年版模版是从中科院模版改编而来,
中科院模版在封面上名字等内容的排列方式需要采用五列表格。这一部分,我也将其重新编写。

随着时代的推进,\LaTeX的各种功能包日渐丰富,很多过去只能从\LaTeXe代码写的功能,
如今可以通过相应的功能包直接实现,在这个模版中,我使用了几个新的功能包,
其中最新的当属刚刚发布的hyperref更新包,增加了hidelinks命令,可以直接将链接的边框去掉,
不用采用将边框颜色设为白色的方式了。

就像\LaTeX的版本总是在接近$\pi$的值一样,这份模版并不是完美的,比如对数学系的定理体系支持不足,
留在以后版本再发布或者请有兴趣的爱好者共同修改。编写这一版本的基本目的是没有任何\LaTeX基础的同学可以比较轻松地利用它给自己的毕业论文排一个满意的版面,整个模版没有留太多选项,
可供修改的选项只有两个:单面双面的选择和链接的颜色的有无。在模版中,我对绝大多数的语句,
都做了中文注释,解释其作用,方便有兴趣的同学研究,我也是一个初学者,作出的这份模版,
我想,应该是比较适合初学者胃口的。

\end{preface}
